\section{Overview}\label{sec:overview}
This section gives an overview on the main features of LOIS, and
how they have been implemented, and how to use them. More details
will be given in the further sections.

\def\lzWhile{{\mbox{\bf while}}\xspace}
\def\lzFor{{\mbox{\bf for}}\xspace}
\def\lzSet{{\mbox{\bf set}}\xspace}
\def\lzInt{{\mbox{\bf int}}\xspace}
\def\lzIn{{\mbox{\bf in}}\xspace}
\def\lzBool{{\mbox{\bf bool}}\xspace}
\def\lzIf{{\mbox{\bf if}}\xspace}

LOIS is based on the hybrid pseudoparallel semantics. The paper
\cite{lois-sem} explains this semantics in detail, and how an
interpreter can be implemented in practice by using \emph{contexts},
which form a stack which changes whenever a local variable is created (possibly by a
\verb-for- loop), or a conditional is used. 
C++ allows a programming technique known as RAII, that is,
automatic initialization and finalization of variables when
a local variable enters or exits the scope. This is exactly
what we need -- our implementation
uses RAII to change the contents of the LOIS stack when \lzIf,
\lzFor, and \lzWhile\ constructs are used.
%For technical reasons, the syntax of our C++ library is a bit
%different than one presented in Section \ref{sec_semantics}.
Note that, since \lzIf\ and \lzWhile\ change the current context,
we were unable to use C++'s \verb-if- and \verb-while- statements directly
-- instead, \verb-If- and \verb-While- macros are used, and
\verb-Ife- for if-then-else.

The type \verb|lset| represents a LOIS set together with its
inner context, and the polymorphic type \verb|elem| represents
a v-element. V-elements can represent 
an integer (type \verb-int-), a term over $\calA$
(type \verb-term-), a pair (type \verb-elpair-), a tuple (type \verb-eltuple-), or a set (type \verb-lset-). Integers, pairs, and tuples are implemented with the corresponding standard C++ types
(\verb|int|, \verb|std::pair| and \verb|std::vector|, respectively); 
and more types can be added by the programmer.
So the programmer can for instance extend 
\verb|elem| to allow a type 
representing lists or trees of elements,
thus allowing sets of type \verb|lset| to store infinite
sets of lists or trees.
It is well known that integers, pairs and tuples can be encoded in 
the set theory (using Kuratowski's definition of pair, for example);
however, allowing to use them directly in our programs greatly improves
both readability and efficiency.

Hybrid pseudoparallel
looping over a set $X$ of type \verb|lset| is done with \verb|for(elem x:X)|.
This is implemented using the C++11 range-based loop. We can
check the specific type of \verb|x|, as well as inspect its
components, with functions such as \verb|is<T>|, \verb|as<T>|
(where \verb-T- is one of the types listed above) and \verb|isSet|, \verb|asSet|.
The syntactic sugar \verb|lsetof<T>| is provided for defining sets which
can only include elements of one specific type \verb-T- -- this
allows static type checking, and eliminates the necessity
of using \verb|is| and \verb|as| functions.

In some cases, such low-level representation of elements is
not enough: for example, consider the function \verb|extract(X)|
which returns the only element of a set $X$ of cardinality 1.
If $X = \{a | a=b\} \cup \{0 | a \neq b\}$, then it is not
possible to represent \verb|extract(X)| as \verb|elem|
in the context $\{a \in \bbA, b \in \bbA\}$, since
each \verb|elem| has to be of specific type, and in our case,
\verb|extract(X)| can be either a term or an integer.
In this case, we can use the type \verb|lelem|, which represents
\emph{piecewise v-elements} (see Section \ref{app_piecewise}) -- that is, ones which may have
different representations depending on the constraints satisfied
by variables in the context. Internally,
this type is represented with a set -- thus, 
\verb|extract(X)| simply wraps the set $X$ into a piecewise
element. Type \verb|lbool| represent a piecewise boolean variable,
which boils down to a formula with free variables from its inner
context.

All the conditions appearing in \verb|If| and \verb|While|
statements are evaluated into first-order formulas over the underlying structure~$\calA$. A solver is used to check whether the set of all
constraints on the stack is satisfiable (and thus whether
to execute a statement or not). Also, a method of simplifying
formulas is necessary, to obtain legible presentations of
results, and to make the execution of the sequel of the program
more effective.

The membership function \verb|memberof(X,Y)|, as well as set equality \verb|X==Y| and
inclusion \verb|subseteq(X,Y)|, have been implemented straightforwardly using
\verb|lbool| and hybrid pseudoparallel iteration over the sets
involved. They are defined with a mutual recursion -- set equality is
a conjunction of two set inclusions, set inclusion $X \subseteq Y$ is
evaluated by looping over all elements of $X$ and checking whether
they are members of $Y$, and membership $x \in Y$ is evaluated by
looping over all elements of $Y$ and checking whether they are
equal to $x$. Equality and relation symbols applied 
to terms result in first order formulas.

\def\LOISzero{LOIS${}^0$\xspace}

To enforce
fully pseudoparallel computation, thus simulating \LOISzero from
\cite{lois-sem}, write \[\verb|for(relem e: fullypseudoparallel(X))|.\]

The underlying structure $\calA$ is not given at the start of the program;
instead, it is possible to define new sorts and new relations during the
execution. Our prototype includes several relations with decidable
theories (order, random partition, random graph, homogeneous tree), as well as solvers for these theories.
Also, it allows consulting external SMT solvers.
