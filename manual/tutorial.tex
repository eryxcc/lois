\vspace{-1em}
\section{Tutorial}\label{sec_tut}

\subsection{Installation}


The current development snapshot of LOIS can be downloaded from GitHub.
LOIS webpage \url{http://www.mimuw.edu.pl/~erykk/lois/} contains the prototype
of LOIS. Note that this prototype is slightly outdated -- this manual
documents the version in development.

LOIS can be installed as follows:

\begin{itemize}
\item You need {\tt gcc} (supporting the C++11 standard), and {\tt make}. Those are very standard tools under Unix-like environments.
\item To try the development snapshot, you need {\tt git}.
Clone the git repository: \\ {\tt git clone https://github.com/eryxcc/lois}
\item To try the prototype instead, you also need {\tt tar}. Download {\tt lois.tgz} and do: {\tt tar zxf lois.tgz}
\item Enter the lois directory created: \\ {\tt cd lois}
\item Run make: {\tt make}
\end{itemize}

The LOIS library is created at {\tt bin/liblois.a}. The headers are 
{\tt include/lois*.h}. We assume that the reader is familiar with the basics
of programming in C++.

\subsection{Writing with LOIS}

We start with some examples of 
simple statements using LOIS (left), 
and their explanations (right). 
The key constructs and features are emphasized in the text.
%A more formal presentation of the syntax is deferred to Appendix~\ref{app_pseudo},
%due to lack of space.

\begin{codecommenttable}
\codecommentrow{initialization}{
To use LOIS, we need to include the appropriate file. LOIS types are in the namespace 
{\tt lois}.
}
\end{codecommenttable}

\begin{codecommenttable}

\codecommentrow{main}
  {\textbf{Initialize} LOIS. Sets can be output in ASCII
  (portable), Unicode (useful for testing) or in the \LaTeX\ markup, which we use in this paper.}

\\ \codecommentrow{domain}
  {Create an infinite \textbf{domain}, named $\bbA$.
  The variable {\tt A} stores its underlying set $A$.
  This outputs $\inco{domain}$.}
	
\\ \codecommentrow{resetX}
    {Create a \textbf{set}~$X$, and set it to $A$.
		 The type {\tt lset} is 
    the type of a set; {\tt l} signifies that we will use it as an lvalue, that is, change its contents.
		Sets can store elements of various types. We \textbf{add} the number $5$ to $X$.
    This outputs $\inco{resetX}$, a shorthand 
		for $\set{b|b\in\bbA}\cup\set5$.}
		
		
\\ \codecommentrow{pairs}{
	The set {\tt Pairs} is a \textbf{typed set},
	whose elements are of the type {\tt elpair},
	denoting pairs; initially it is empty.
	\textbf{Iterate} over all pairs of elements of $A$,
and construct the set of all ordered \textbf{pairs}.
The output
is $\inco{pairs}$.}		

\\ \codecommentrow{linear}{
	By default,
		$\bbA$ is equipped with a \textbf{linear order}~$\le$,
	isomorphic to the rational numbers. 
We test that the order is transitive,
by setting a \textbf{boolean} flag of type {\tt lbool}
whenever a counterexample is found. 
 Note that we use a special \textbf{conditional} {\tt If} with LOIS, actually a macro.
This outputs \emph{\inco{linear}}.}

% \\ \codecommentrow{set3}{
% As in mathematics, we can  \textbf{iterate} through sets
% constructed previously, and execute
% \textbf{functions}  inside the loop.
% To apply the {\tt interval} function (defined below),
% we \textbf{cast} the two elements of the pair to the type {\tt term}, representing elements of a domain.
% % Finally, infinite sets can be \textbf{nested}.
% We construct the set of all open intervals in $\bbA$.
% % Note that we use a special LOIS conditional \verb-If-
% % for working with LOIS~types.
% The result is $\inco{set3}$.}

% \\ \codecommentrow{interval}{
%   % The \emph{interval} function is defined outside
%   %  the body of the main function.
%    Function arguments and  return values are \textbf{rvalues},
%     for sets this is signified by the letter {\tt r}
%    in the type {\tt rset}.
%    {\tt FILTER} is a macro for
% \textbf{selecting} a subset with a given property;
% here -- the open interval in $A$ with endpoints $a,b$.}
\end{codecommenttable}



Now we demonstrate an algorithm 
manipulating on infinite sets, namely the reachability 
algorithm for infinite graphs. Reachability is very important
for applications of LOIS~\cite{lois-sat}.
%, which was used in Example~\ref{ex:minimal}. 
%This has applications
%in verification, which we discuss in Section~\ref{sec_applications}.

\begin{codecommenttable}	
\codecommentrow{graph}{
	First, let's define some infinite directed graph.
	In this case, the vertices are pairs of elements of $A$,
	and the edgeset is $E$.
	This outputs~$\inco{graph}$.
	Therefore, edges are of the form $(p,q)\rightarrow (q,r)$,
	where $p\neq r$.
}

% \\ \codecommentrow{declareatom}{
%   We \textbf{declare} $a,b$ to be elements of $\bbA$,
%   and \textbf{assert} they are distinct. We want to determine whether $t=(b,a)$
%   is reachable from $s=(a,b)$ in the graph defined by $E$.
% }

\\ \codecommentrow{test_reach}{
For each pair $(a,b)$ of distinct elements of $A$,
	we test reachability of $(b,a)$ from $(a,b)$ in our graph,
  using the function $\tt reach$ given below
  (and defined outside of the body
  of the main function).
  Note that we use \textbf{local variables} $s$ and $t$.
  % The flag $\tt reached$ is of type $\tt lbool$,
  % which is the type used for (lvalue) \textbf{booleans} in LOIS.
  The output is~$\inco{test_reach}$.
An example (shortest) path
	is $(a,b)\rightarrow (b,c)\rightarrow (c,d) \rightarrow(d,b)\rightarrow (b,a)$,
	where $c,d$ are distinct from $a,b$.}

\\ \codecommentrow{reachability}{
	We use a \textbf{fixpoint algorithm} to 
  compute the vertices reachable from $S$ in a graph with edgeset $E$.
  Special  \textbf{while loops} are used with LOIS conditionals
  (again, macros). 
	Correctness of this algorithm is clear.
  One could also use a BFS traversal,
  however, the presented algorithm is slightly easier to analyse.
   Termination (in finite time) can be proved for 	 
   \emph{any} graph
	defined in LOIS from the domain $\bbA$,
  using equality only \cite{lois-sat}.
%(see Section~\ref{sec_applications}).
}


\end{codecommenttable}
\vspace{-1em}\noindent
The intuitive semantics of the above examples is clear,
as they follow closely the set-builder notation -- we iterate through
some set and collect the results in some resulting set
(an exception the use of the boolean flag).
In general, our novel \emph{pseudo-parallel} semantics 
is meaningful also when  other operations are performed 
inside the loops, e.g. removing elements from a set, 
or declaring local variables
within the body of the loop. This requires extending the set-builder intuitions, and defining a proper semantics 
\cite{lois-sem}.

% Here is an even less intuitive example.
%
% \\ \codecommentrow{count}
% 	What should the result of this loop be?
% 	The output is $\inco{count}$.
% }

