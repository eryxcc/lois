% \section{Prover details}
%
% We have been trying to use external provers for these problems.
% The current prototype is able to export queries (empty theory only) to a
% smtlib-compliant SMT solver; it is not able to apply simplification tactics yet.
% We have tried using Yices \cite{yices}, CVC3 \cite{cvc3}, and Z3 \cite{z3},
% but our experiments so far did not yield satisfactory results:
% it seems that SMT solvers are not optimized for the kinds of formulas which are
% naturally generated
% by LOIS programs (first order formulas with many quantifiers), although this could
% be also caused by our lack of experience with SMT solvers.
% There are also first order provers based on the supposition calculus, such
% as SPASS \cite{spass}; however, SPASS did not yield satisfactory results either,
% and it does not seem to solve the problem of simplification.
% Another potential approach is to use QBF solvers -- but transforming a first order formula to QBF
% requires some work, and this does not solve the problem of simplification either.
%
% Instead, LOIS currently implements several simple techniques for simplification
% of formulas (eliminate quantifier when the value is known, eliminate repeating
% subformulas, etc.), and checks satisfiability by looping over all the possible
% assignments of variables, using the algorithm explained in Section
% \ref{sec_symbols}
% (with some optimizations --- that is, if the set of variables can
% be split into two subsets $V_1$ and $V_2$ such that we never compare variables
% in $V_1$ to variables in $V_2$, thet set of possible assignments is generated
% independently in $V_1$ and $V_2$ --- and relation is not generated for $V_i$
% if it is never used in this set). This is also occassionally done to simplify formulas (all the
% possible valuations are checked, and subformulas are replaced with {\tt true}
% or {\tt false} if a given subformula was always given the same result).
%

\section{Safety of programming}\label{subsecsafety}

It is possible to create a LOIS program which compiles and terminates without
a run-time exception, but nonetheless works incorrectly; for example, 
the following function will count the cardinality of the set incorrectly,
due to using the type \verb-int- instead of the piecewise integer \verb-lnum<int>-.

\incc{cardinalitybad}

Of course, this problem cannot be completely solved---it is possible to create an
incorrect C++ program even without LOIS.
However, the programmer should be mostly safe, if they observe the following
rules.

\begin{itemize}
\item The lvalue types ({\tt lbool}, {\tt lset}, {\tt lnum}, etc.) are used
for all the local (and global) variables, except the iterators for looping over
sets and 
\verb-setof<T>-'s, for which
\verb-elem- and \verb-T- (or \verb-auto-) are used respectively, and temporary values
(function
parameters, function return values), for which rvalue types may (and should) be used.
The difference between lvalue and rvalue types is technical---to handle
assignments and other changes correctly, lvalue types require extra
information (the inner context), while rvalue types do not.
We have decided to use two distinct types---the rationale is that 
a lvalue variable should correspond to a specific stack (context), and the stack
changes during a function call or return;
%(this rationale is not very
%strong, though, since only $V_S$ is important---not the whole stack---and
%$V_S$ does not change during function calls and returns)
also avoiding this
extra information could improve the efficiency.
Assigning a rvalue to a lvalue (or, in general, modifying a lvalue in any way
which takes a rvalue as a parameter) is legal only if the rvalue does not
depend on the sort variables which have been introduced since the
lvalue was created. Thus, the following implementation of
{\tt max} would throw an exception for some sets $X$:
\incc{maxbad}

\item Non-structural programming constructs, such as {\tt continue},
{\tt break}, {\tt return}, and {\tt goto} are considered harmful and
should be avoided. Suppose a function is called with the current context $C$,
then new constaints are pushed on the stack, changing the current context
to $C'$. Now, the {\tt return} statement is used. This will break the execution flow
not only for all the $C'$-valuations, but actually 
for all the $C$-valuations, irregardless of whether they are $C'$-valuations
or not (the necessary condition for executing 
the return statement is that $\val{C'}$ is non-empty).
Moreover, {\tt If} statements could execute both branches, and they are
internally implemented with a {\tt for} loop, which might make
the behavior of {\tt continue} and {\tt break} different than expected.
Thus, the only way of using the {\tt return} statement which is not considered
harmful is to  declare a local lvalue as the first statement (before any loops), 
modify it in the body of the function,
and then return its value as the last statement of the function.
Note that this approach is similar to the one used in original Pascal.
\end{itemize}


