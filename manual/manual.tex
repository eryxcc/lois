\documentclass[a4paper]{article}
% \documentclass[a4paper,USenglish]{lipics}
%\usepackage[latin2]{inputenc}

%\def\dblind#1{#1}
\usepackage{amsthm}
\usepackage{amsmath}
\usepackage{amssymb}
%\usepackage{comment}
%\usepackage{enumitem}
%\usepackage{tcolorbox}
%\usepackage{graphicx}

%\usepackage{cutwin}


\usepackage{color}
\usepackage{colortbl}
\usepackage{longtable}
\usepackage{hyperref}

\usepackage{listings}

\usepackage{wasysym}
%\usepackage{lipicsfix}
\usepackage{xspace}
%\usepackage{wrapfig}

%\usepackage{algorithm2e}

\def\fraisse{Fraiss\'e\xspace}
\def\stack{{{\textit{Current}}}}
\def\bra#1{{$\langle$\tt #1$\rangle$}}
\def\set#1{\{#1\}}
\def\pseudo{{\pluto}}
\def\com#1{\qquad//{\emph{#1}}}

\def\function{{\mbox{\bf function}}}
\def\for{{\mbox{\bf for}}}
\def\while{{\mbox{\bf while}}}
\def\pcif{{\mbox{\bf if}}}
\def\pcreturn{{\mbox{\bf return}}}
\def\pcor{{\mbox{ \bf or }}}
\def\pcand{{\mbox{ \bf and }}}
\def\pctrue{{\mbox{\bf true}}}
\def\pcfalse{{\mbox{\bf false}}}
\def\addel{{\mbox{{\tt +=}}}}
\def\remel{{\mbox{{\tt -=}}}}
\def\i{\hskip 3em}
\def\val#1{{\mbox{Val}(#1)}}
% \def\varext#1#2{{{{\mbox{Val}}_{#1}}(#2)}}
\def\Var{{\mathcal V}}
\newcommand{\st}{\ |\ }
% \DeclareMathOperator{\constr}{Definaormulas 
\definecolor{mygreen}{rgb}{0,0.6,0}
 \definecolor{mygray}{rgb}{0.5,0.5,0.5}
 \definecolor{mymauve}{rgb}{0.58,0,0.82}
\definecolor{mygray}{rgb}{0.95,0.95,0.95}

% \mainmatter              % start of the contributions
%
\title{LOIS: technical documentation}
%
%\titlerunning{LOIS: technical documentation}  % abbreviated title (for running head)
%                                     also used for the TOC unless
%                                     \toctitle is used
%
\author{Eryk Kopczy\'nski \and Szymon Toru\'nczyk}
%\authorrunning{\dblind{Eryk Kopczy\'nski and Szymon Toru\'nczyk}} % abbreviated author list (for running head)

%%% list of authors for the TOC (use if author list has to be modified)
%\tocauthor{\dblind{Eryk Kopczy\'nski, Szymon Toru\'nczyk}}

%\dblind{University of Warsaw, Poland}\\
% \dblind{\email{\{erykk,szymtor\}@mimuw.edu.pl}}




% \author{Eryk~Kopczy\'nski}%\thanks{Supported by ?}}
% \author{Szymon~Toru\'nczyk}%\thanks{Supported by ?}}
%
% \affil{University of Warsaw\\
%   \texttt{\{erykk,szymtor\}@mimuw.edu.pl}}
%
%
% %\institute{Institute of Informatics, University of Warsaw\\
% %  \texttt{erykk@mimuw.edu.pl}
% %}
%
% %\title{LOIS: a Practical C++ Library for Handling Infinite Sets}
% \title{LOIS: Handling Infinite Sets in Theory and Practice}

% the basic fields:
\def\bbA{{\mathbb A}} % arbitrary universe
\def\bbB{{\mathbb B}} % sub-universe
\def\bbN{{\mathbb N}} % integer numbers
 \def\bbZ{{\mathbb Z}} % integer numbers
\def\bbQ{{\mathbb Q}} % rational numbers
\def\bbP{{\mathbb P}} % non-negative real numbers
\def\bbR{{\mathbb R}} % real numbers
\def\bbF{{\mathbb F}} % any field
\def\calA{{\mathcal A}}
\def\calB{{\mathcal B}}
\def\calC{{\mathcal C}}
\def\calF{{\mathcal F}}
\def\calZ{{\mathcal Z}}
\def\calQ{{\mathcal Q}}
 \def\calR{{\mathcal R}}
\def\rado{{\mathcal G}}
\def\calT{{\mathcal T}}
\def\aut#1{{\mathrm Aut(#1)}}

\def\myparagraph#1{\smallskip{\noindent\bf #1.}}
\newcounter{mycount}[section]
\renewcommand{\themycount}{\thesection.\arabic{mycount}}

%\newtheorem{theorem}[mycount]{Theorem}
%\newtheorem{lemma}[mycount]{Lemma}
%\newtheorem{definition}[mycount]{Definition}
% \newtheorem{proposition}[mycount]{Proposition}
%\newtheorem{remark}[mycount]{Remark}

% \theoremstyle{plain}

\newenvironment{runex}
{\smallskip\noindent\begin{lrbox}{0}%
\begin{minipage}{\textwidth}%
}
{\end{minipage}\end{lrbox}\colorbox{mygray}{\noindent\usebox{0}}\smallskip}

% \newenvironment{runex}
% {
% % \noindent\rule{\textwidth}{0.4pt}\\
% \colorbox{mygray}
% \begingroup\small\emph{Running example.}
% }
% {\endgroup\\
% % \noindent\rule{\textwidth}{0.4pt}
% }

% \newenvironment{runex}
% {}
% {}
 
% \newenvironment{runex}
% {\begin{tcolorbox}[oversize, colframe = mygray,  colback  = mygray]}
% {\end{tcolorbox}}


% \newtcolorbox{runex}
% {
%   colframe = mygray,
%   colback  = mygray,
% 	oversize
%   % coltitle = black,
%   % title    = Running example,
% }


%\newtheorem{problem}[mycount]{Problem}
%\newtheorem{warunek}[mycount]{Condition}
%\newtheorem{corol}[mycount]{Corollary}

\lstdefinestyle{customc}{
  belowcaptionskip=1\baselineskip,
  breaklines=true,
% frame=L,
  xleftmargin=\parindent,
  language=C++,
  showstringspaces=false,
  basicstyle=\scriptsize\ttfamily,
  keywordstyle={\bf},
  commentstyle=\itshape
% identifierstyle=\color{blue},
% stringstyle=\color{red},
}
\lstset{language=C++,style=customc}

%\newenvironment{proof}[1][]%
%{\medskip{\bf Proof #1 }}%
%{}
%\def\qed{\hfill$\rule{2mm}{2mm}$\par\medskip}

\def\inctut#1{
\begin{wrapfigure}{l}{0.6\textwidth}
  \begin{center}
\lstinputlisting{snippets/snippet-#1}
\end{center}
\end{wrapfigure}}
\def\incc#1{\vspace{-0em}\lstinputlisting{snippets/snippet-#1}\vspace{-0em}}
\def\inco#1{\input{snippets/output-#1}\unskip}
\def\pspace{\textsc{PSpace}}
\def\ptime{\textsc{PTime}}
% \newenvironment{codecomment}[1]
% {\noindent\fcolorbox{black}{mygray}{\begin{minipage}{0.4\textwidth}\incc{#1}\end{minipage}}
% \begin{minipage}{0.6\textwidth}
% }
% {\end{minipage}}
% \newenvironment{codecomment}[1]
% {\noindent\begin{tabular}{p{0.4\textwidth}|@{\ }p{0.6\textwidth}}%
%  % \incc{#1}&%
% \fcolorbox{cyan}{yellow}{\parbox{\dimexpr\linewidth-2\fboxsep-2\fboxrule\relax}{\incc{#1}}}&
% {\renewcommand{\arraystretch}{1.3}%
\newenvironment{codecommenttable}
{\begin{longtable}{>{\columncolor[gray]{0.95}}p{0.4\textwidth}@{\ \ }p{0.6\textwidth}}%
}
{\end{longtable}}

\newtheorem{example}{Example}

\newcommand{\codecommentrow}[2]
{
	% \fcolorbox{white}{white}
		{\parbox{\dimexpr\linewidth-2\fboxsep-2\fboxrule\relax}{%
		% \lstset{aboveskip=-1em,belowskip=-1.5em}%
		\incc{#1}}}
		&%
	% \fcolorbox{white}{white}%
	{%
		\parbox%
		{\dimexpr\linewidth-2\fboxsep-2\fboxrule\relax}%
		{\vspace{-0.em}#2\vspace{-0.em}}%
	}
}




% \newenvironment{codecomment}[1]
% {\begin{codecommenttable}
% 	\codecommentrow{#1}\bgroup
% }
% {\egroup
% \end{codecommenttable}
% }

\begin{document}
 
\maketitle

\begin{abstract}
LOIS is a C++ library allowing iterating through certain infinite sets, in finite time.
The resulting language has an intuitive semantics,
corresponding to execution of infinitely 
many threads in parallel.
This allows to merge the power of abstract mathematical constructions 
into imperative programming.
Infinite sets are internally represented 
using first order formulas over some underlying logical structure. 
To effectively handle such sets, we use and implement SMT solvers for various first order
theories. LOIS has applications in education, and in verification of infinite state systems.

This is a technical documentation of LOIS, describing how to write programs using it.
\end{abstract}

% \keywords{infinite sets, automata theory, sets with atoms, pseudo-variables, pseudo-parallel computation}

%\setcounter{tocdepth}{2}
%\makeatletter
%\newcommand*{\toccontents}{\@starttoc{toc}}
%\renewcommand*\l@author[2]{} \renewcommand*\l@title[2]{}
%\def\authcount#1{}
%\makeatother
%\toccontents
%\newpage

\vspace{-2em}
\section{Introduction}\label{sec_intro}

LOIS (Looping Over Infinite Sets) is a C++ library which allows working on 
\emph{definable} infinite sets in a natural way. We can create an infinite
\emph{domain}, let's say $\bbA$, possibly with some relational and functional symbols, 
and then use the \emph{pseudo-parallel} semantics to iterate over it in a natural way. 
This gives us new sets, for example $\{(x,y): x \in \bbA, y \in \bbA, x \neq y\}$,
which can be iterated over in turn, or checked for emptiness. A LOIS program will
work in finite time as long as the first order theory of $\bbA$ is decidable.
LOIS is open source, released under MIT license.

Its homepage is at
{\tt http://www.mimuw.edu.pl/\textasciitilde erykk/lois/},
and its GitHub repository is at
{\tt https://www.github.com/eryxcc/lois}.


This document is a technical description of LOIS, and thus, many theoretical details
have been omitted. See the papers on the website above for details about:

\begin{itemize}
\item the theoretical foundations: definable sets, homogeneous structures, \cite{lois-sat}
\item how the solvers (built-in, CVC4, Z3, SPASS) are used to make the computation
possible, \cite{lois-sat}
\item the results of our tests of internal and external solvers, \cite{lois-sat}
\item applications, \cite{lois-sat}
\item the novel pseudoparallel semantics which LOIS is using to handle the infinite sets.
\cite{lois-sem}
\end{itemize}

Note that, in the current prototype, all the core functionality works, but in case of
some functions, variants which accept more complex types (say, {\tt lnumof<T>} or {\tt
lsetof<T>} instead of the basic {\tt elem} or {\tt lset}) may be still missing,
or some obvious type casting might be missing, making C++ unable to guess
what typecasts should be used.
Usually
such more complex variants should be easy to write.

LOIS has been tested on machines with the following configurations:
\begin{itemize}
\item gcc version 4.4.3, architecture i686, Ubuntu Linux
\item gcc version 4.6.3, architecture x86\_64, Ubuntu Linux
\item gcc version 4.7.3, architecture x86\_64, PLD Linux
\end{itemize}

\subsection{Changelog}
On May 28, 2017 LOIS has been changed to use static typing by default: instead of using
the polymorphic element type \verb|elem|, and \verb|lset| which can include elements of
any type, the basic types is now \verb|lsetof<T>|. The type \verb|elem| (explicitly created
with the function \verb|elof|) and \verb|lset| (defined as \verb|lsetof<elem>|) are
available in \verb|lois-weak.h|. {\bf THE MANUAL HAS NOT BEEN UPDATED YET} except
the automatically generated code snippets.

\vspace{-1em}
\section{Tutorial}\label{sec_tut}

\subsection{Installation}


The current development snapshot of LOIS can be downloaded from GitHub.
LOIS webpage \url{http://www.mimuw.edu.pl/~erykk/lois/} contains the prototype
of LOIS. Note that this prototype is slightly outdated -- this manual
documents the version in development.

LOIS can be installed as follows:

\begin{itemize}
\item You need {\tt gcc} (supporting the C++11 standard), and {\tt make}. Those are very standard tools under Unix-like environments.
\item To try the development snapshot, you need {\tt git}.
Clone the git repository: \\ {\tt git clone https://github.com/eryxcc/lois}
\item To try the prototype instead, you also need {\tt tar}. Download {\tt lois.tgz} and do: {\tt tar zxf lois.tgz}
\item Enter the lois directory created: \\ {\tt cd lois}
\item Run make: {\tt make}
\end{itemize}

The LOIS library is created at {\tt bin/liblois.a}. The headers are 
{\tt include/lois*.h}. We assume that the reader is familiar with the basics
of programming in C++.

\subsection{Writing with LOIS}

We start with some examples of 
simple statements using LOIS (left), 
and their explanations (right). 
The key constructs and features are emphasized in the text.
%A more formal presentation of the syntax is deferred to Appendix~\ref{app_pseudo},
%due to lack of space.

\begin{codecommenttable}
\codecommentrow{initialization}{
To use LOIS, we need to include the appropriate file. LOIS types are in the namespace 
{\tt lois}.
}
\end{codecommenttable}

\begin{codecommenttable}

\codecommentrow{main}
  {\textbf{Initialize} LOIS. Sets can be output in ASCII
  (portable), Unicode (useful for testing) or in the \LaTeX\ markup, which we use in this paper.}

\\ \codecommentrow{domain}
  {Create an infinite \textbf{domain}, named $\bbA$.
  The variable {\tt A} stores its underlying set $A$.
  This outputs $\inco{domain}$.}
	
\\ \codecommentrow{resetX}
    {Create a \textbf{set}~$X$, and set it to $A$.
		 The type {\tt lset} is 
    the type of a set; {\tt l} signifies that we will use it as an lvalue, that is, change its contents.
		Sets can store elements of various types. We \textbf{add} the number $5$ to $X$.
    This outputs $\inco{resetX}$, a shorthand 
		for $\set{b|b\in\bbA}\cup\set5$.}
		
		
\\ \codecommentrow{pairs}{
	The set {\tt Pairs} is a \textbf{typed set},
	whose elements are of the type {\tt elpair},
	denoting pairs; initially it is empty.
	\textbf{Iterate} over all pairs of elements of $A$,
and construct the set of all ordered \textbf{pairs}.
The output
is $\inco{pairs}$.}		

\\ \codecommentrow{linear}{
	By default,
		$\bbA$ is equipped with a \textbf{linear order}~$\le$,
	isomorphic to the rational numbers. 
We test that the order is transitive,
by setting a \textbf{boolean} flag of type {\tt lbool}
whenever a counterexample is found. 
 Note that we use a special \textbf{conditional} {\tt If} with LOIS, actually a macro.
This outputs \emph{\inco{linear}}.}

% \\ \codecommentrow{set3}{
% As in mathematics, we can  \textbf{iterate} through sets
% constructed previously, and execute
% \textbf{functions}  inside the loop.
% To apply the {\tt interval} function (defined below),
% we \textbf{cast} the two elements of the pair to the type {\tt term}, representing elements of a domain.
% % Finally, infinite sets can be \textbf{nested}.
% We construct the set of all open intervals in $\bbA$.
% % Note that we use a special LOIS conditional \verb-If-
% % for working with LOIS~types.
% The result is $\inco{set3}$.}

% \\ \codecommentrow{interval}{
%   % The \emph{interval} function is defined outside
%   %  the body of the main function.
%    Function arguments and  return values are \textbf{rvalues},
%     for sets this is signified by the letter {\tt r}
%    in the type {\tt rset}.
%    {\tt FILTER} is a macro for
% \textbf{selecting} a subset with a given property;
% here -- the open interval in $A$ with endpoints $a,b$.}
\end{codecommenttable}



Now we demonstrate an algorithm 
manipulating on infinite sets, namely the reachability 
algorithm for infinite graphs. Reachability is very important
for applications of LOIS~\cite{lois-sat}.
%, which was used in Example~\ref{ex:minimal}. 
%This has applications
%in verification, which we discuss in Section~\ref{sec_applications}.

\begin{codecommenttable}	
\codecommentrow{graph}{
	First, let's define some infinite directed graph.
	In this case, the vertices are pairs of elements of $A$,
	and the edgeset is $E$.
	This outputs~$\inco{graph}$.
	Therefore, edges are of the form $(p,q)\rightarrow (q,r)$,
	where $p\neq r$.
}

% \\ \codecommentrow{declareatom}{
%   We \textbf{declare} $a,b$ to be elements of $\bbA$,
%   and \textbf{assert} they are distinct. We want to determine whether $t=(b,a)$
%   is reachable from $s=(a,b)$ in the graph defined by $E$.
% }

\\ \codecommentrow{test_reach}{
For each pair $(a,b)$ of distinct elements of $A$,
	we test reachability of $(b,a)$ from $(a,b)$ in our graph,
  using the function $\tt reach$ given below
  (and defined outside of the body
  of the main function).
  Note that we use \textbf{local variables} $s$ and $t$.
  % The flag $\tt reached$ is of type $\tt lbool$,
  % which is the type used for (lvalue) \textbf{booleans} in LOIS.
  The output is~$\inco{test_reach}$.
An example (shortest) path
	is $(a,b)\rightarrow (b,c)\rightarrow (c,d) \rightarrow(d,b)\rightarrow (b,a)$,
	where $c,d$ are distinct from $a,b$.}

\\ \codecommentrow{reachability}{
	We use a \textbf{fixpoint algorithm} to 
  compute the vertices reachable from $S$ in a graph with edgeset $E$.
  Special  \textbf{while loops} are used with LOIS conditionals
  (again, macros). 
	Correctness of this algorithm is clear.
  One could also use a BFS traversal,
  however, the presented algorithm is slightly easier to analyse.
   Termination (in finite time) can be proved for 	 
   \emph{any} graph
	defined in LOIS from the domain $\bbA$,
  using equality only \cite{lois-sat}.
%(see Section~\ref{sec_applications}).
}


\end{codecommenttable}
\vspace{-1em}\noindent
The intuitive semantics of the above examples is clear,
as they follow closely the set-builder notation -- we iterate through
some set and collect the results in some resulting set
(an exception the use of the boolean flag).
In general, our novel \emph{pseudo-parallel} semantics 
is meaningful also when  other operations are performed 
inside the loops, e.g. removing elements from a set, 
or declaring local variables
within the body of the loop. This requires extending the set-builder intuitions, and defining a proper semantics 
\cite{lois-sem}.

% Here is an even less intuitive example.
%
% \\ \codecommentrow{count}
% 	What should the result of this loop be?
% 	The output is $\inco{count}$.
% }


\section{Overview}\label{sec:overview}
This section gives an overview on the main features of LOIS, and
how they have been implemented, and how to use them. More details
will be given in the further sections.

\def\lzWhile{{\mbox{\bf while}}\xspace}
\def\lzFor{{\mbox{\bf for}}\xspace}
\def\lzSet{{\mbox{\bf set}}\xspace}
\def\lzInt{{\mbox{\bf int}}\xspace}
\def\lzIn{{\mbox{\bf in}}\xspace}
\def\lzBool{{\mbox{\bf bool}}\xspace}
\def\lzIf{{\mbox{\bf if}}\xspace}

LOIS is based on the hybrid pseudoparallel semantics. The paper
\cite{lois-sem} explains this semantics in detail, and how an
interpreter can be implemented in practice by using \emph{contexts},
which form a stack which changes whenever a local variable is created (possibly by a
\verb-for- loop), or a conditional is used. 
C++ allows a programming technique known as RAII, that is,
automatic initialization and finalization of variables when
a local variable enters or exits the scope. This is exactly
what we need -- our implementation
uses RAII to change the contents of the LOIS stack when \lzIf,
\lzFor, and \lzWhile\ constructs are used.
%For technical reasons, the syntax of our C++ library is a bit
%different than one presented in Section \ref{sec_semantics}.
Note that, since \lzIf\ and \lzWhile\ change the current context,
we were unable to use C++'s \verb-if- and \verb-while- statements directly
-- instead, \verb-If- and \verb-While- macros are used, and
\verb-Ife- for if-then-else.

The type \verb|lset| represents a LOIS set together with its
inner context, and the polymorphic type \verb|elem| represents
a v-element. V-elements can represent 
an integer (type \verb-int-), a term over $\calA$
(type \verb-term-), a pair (type \verb-elpair-), a tuple (type \verb-eltuple-), or a set (type \verb-lset-). Integers, pairs, and tuples are implemented with the corresponding standard C++ types
(\verb|int|, \verb|std::pair| and \verb|std::vector|, respectively); 
and more types can be added by the programmer.
So the programmer can for instance extend 
\verb|elem| to allow a type 
representing lists or trees of elements,
thus allowing sets of type \verb|lset| to store infinite
sets of lists or trees.
It is well known that integers, pairs and tuples can be encoded in 
the set theory (using Kuratowski's definition of pair, for example);
however, allowing to use them directly in our programs greatly improves
both readability and efficiency.

Hybrid pseudoparallel
looping over a set $X$ of type \verb|lset| is done with \verb|for(elem x:X)|.
This is implemented using the C++11 range-based loop. We can
check the specific type of \verb|x|, as well as inspect its
components, with functions such as \verb|is<T>|, \verb|as<T>|
(where \verb-T- is one of the types listed above) and \verb|isSet|, \verb|asSet|.
The syntactic sugar \verb|lsetof<T>| is provided for defining sets which
can only include elements of one specific type \verb-T- -- this
allows static type checking, and eliminates the necessity
of using \verb|is| and \verb|as| functions.

In some cases, such low-level representation of elements is
not enough: for example, consider the function \verb|extract(X)|
which returns the only element of a set $X$ of cardinality 1.
If $X = \{a | a=b\} \cup \{0 | a \neq b\}$, then it is not
possible to represent \verb|extract(X)| as \verb|elem|
in the context $\{a \in \bbA, b \in \bbA\}$, since
each \verb|elem| has to be of specific type, and in our case,
\verb|extract(X)| can be either a term or an integer.
In this case, we can use the type \verb|lelem|, which represents
\emph{piecewise v-elements} (see Section \ref{app_piecewise}) -- that is, ones which may have
different representations depending on the constraints satisfied
by variables in the context. Internally,
this type is represented with a set -- thus, 
\verb|extract(X)| simply wraps the set $X$ into a piecewise
element. Type \verb|lbool| represent a piecewise boolean variable,
which boils down to a formula with free variables from its inner
context.

All the conditions appearing in \verb|If| and \verb|While|
statements are evaluated into first-order formulas over the underlying structure~$\calA$. A solver is used to check whether the set of all
constraints on the stack is satisfiable (and thus whether
to execute a statement or not). Also, a method of simplifying
formulas is necessary, to obtain legible presentations of
results, and to make the execution of the sequel of the program
more effective.

The membership function \verb|memberof(X,Y)|, as well as set equality \verb|X==Y| and
inclusion \verb|subseteq(X,Y)|, have been implemented straightforwardly using
\verb|lbool| and hybrid pseudoparallel iteration over the sets
involved. They are defined with a mutual recursion -- set equality is
a conjunction of two set inclusions, set inclusion $X \subseteq Y$ is
evaluated by looping over all elements of $X$ and checking whether
they are members of $Y$, and membership $x \in Y$ is evaluated by
looping over all elements of $Y$ and checking whether they are
equal to $x$. Equality and relation symbols applied 
to terms result in first order formulas.

\def\LOISzero{LOIS${}^0$\xspace}

To enforce
fully pseudoparallel computation, thus simulating \LOISzero from
\cite{lois-sem}, write \[\verb|for(relem e: fullypseudoparallel(X))|.\]

The underlying structure $\calA$ is not given at the start of the program;
instead, it is possible to define new sorts and new relations during the
execution. Our prototype includes several relations with decidable
theories (order, random partition, random graph, homogeneous tree), as well as solvers for these theories.
Also, it allows consulting external SMT solvers.


\section{Syntax and features}\label{app_pseudo}
In this appendix, we list the key constructions available in LOIS. 
% Some of these constructions have
% already appeared in Section~\ref{sec_tut}, but
% we repeat them here for completeness and a more
% structured presentation.
Also, we 
give an informal description of some constructs, 
which is sufficient for understanding basic programs in LOIS
on an intuitive level. This should not be confused with the formal semantics, which is given in 
\cite{lois-sem}.

\subsection{Types}
LOIS allows the programmer to manipulate sets, which are represented by the type {\tt lset}.


%The distinction between lvalues and rvalues is for technical reasons (see Section~\ref{sec_semantics}).

 There is a polymorphic type {\tt elem} representing elements of sets, which should be used as the type of a control variable in a {\tt for} loop. 
 The programmer can define set elements basing on any C++ type which has the following basic operations
defined: variable substitution (create a copy with the variable changed),
output (to a C++ stream),
equality test, and checking whether it depends on a given variable
(required for optimization). Integers, pairs (type {\tt elpair}), and tuples (type {\tt eltuple}) are based on the standard C++ types, and are defined as  {\tt int}, {\tt pair<elem, elem>} and {\tt vector<elem>}, respectively. 
Terms are represented by the type {\tt term}.

The type {\tt lelem} represents element of a set. 
While {\tt elem} contain elements of a fixed type (e.g., a pair), {\tt lelem} 
represents {\it piecewise elements}, whose type may depend on the variables
in the context -- intuitively,
their type may different in different pseudoparallel threads.
They are described in detail in Section \ref{app_piecewise} below. Also, {\tt lnum<T>}
is a wrapper for piecewise numbers -- they are described in
Subsection \ref{app_piecewise_num} below.

Finally, there are two types for representing booleans whose values may depend on
variables in their current (inner) context, {\tt lbool} and {\tt rbool}. 
Type {tt rbool} simply represents a formula, and {\tt lbool} has an inner context,
allowing it to work according to the pseudoparallel semantics.

\myparagraph{Conversions}
There is an assignment operator for assigning a variable of type {\tt relem} to a 
variable of type {\tt lelem}. Thanks to this and the C++ conversion mechanism, whenever
a variable of type {\tt lelem} is used in place where a variable of type {\tt relem} is
expected, an automatic conversion is performed by the compiler (and similarly for the
other pairs of lvalue and rvalue types).
To convert a {\tt relem e} to a {\tt elem}, \verb-for(elem x: newSet(e))- can be used.
Also, pairs, tuples, terms, integers, and sets are automatically converted to {\tt elem}s.
A variable {\tt x} of type {\tt elem} can be cast to a set, a pair, a tuple, a term, or an integer, using the operations {\tt asSet(x)}, {\tt as<elpair>(x)}, {\tt as<eltuple>(x)}, {\tt as<term>(x)}, and {\tt as<int>(x)}, respectively. It can be also tested whether {\tt x} is a pair, by the operation {\tt is<elpair>(x)} (and similarly for terms, tuples and integers) which returns a value of type {\tt bool}. 

\myparagraph{Sets with type checking}
The types \verb-lsetof<T>- is used instead of \verb-lset- 
to create sets whose all elements are of type \verb-T- (typically, 
\verb-term-, \verb-elpair-, \verb-eltuple-, \verb-lset-, \verb-int-, or \verb-lsetof<U>-). 
This is a wrapper around \verb-lset- which allows only adding elements
of type \verb-T-; moreover, the \verb-for- loop iterates over type \verb-T- instead
of the polymorphic \verb-elem-. 
% For example: \incc{setof}
This improves readability and provides type checking.


% \subsection{Syntax}

\subsection{Flow control}
LOIS has the following constructs constructs for flow control:
\begin{itemize}
\item The conditional {\tt If (cond) I}, where {\tt cond} is a condition of type {\tt rbool} and {\tt I} is an instruction (the body) which is to be executed if the condition is satisfied. There is also a variant 
{\tt Ife (cond) I else J}, where the instruction~{\tt J} is to be executed if the condition fails. {\tt If} and {\tt While} are macros; they should be used with LOIS conditions, rather than the normal {\pcif} and {\while}.
\item The looping construct {\tt While (cond) I}, where {\tt cond} and {\tt I} are as above.
\item The \emph{hybrid pseudoparallel} looping construct {\tt for (elem x:X) I},
where {\tt x} is the name of the introduced {control variable}, {\tt X} is the set (of type {\tt lset}) over which it ranges, and {\tt I} is an instruction (the body of the loop).
\item The fully pseudoparallel looping can be achieved in LOIS using the construct
{\tt for(relem e: fullypseudoparallel(X))}.
\end{itemize}

Furthermore, functions and recursion from C++ can be used. 
Note that since the {\tt for} loop in LOIS is
defined in a hybrid way \cite{lois-sem}, using the {\tt return}, {\tt break},
or {\tt continue} statement inside a loop will cause LOIS to stop processing the
set, and thus unintuitive behavior. The recommended approach is to create a value
(say, {\tt lelem ret}) to represent the returned value, set its value in the loop,
and then {\tt return ret}) after the loop ends.

\subsection{Operations} 
The basic operations on sets (\verb-lset-) are defined as follows. 
In the list, we denote sets (\verb-lset-) with \verb-X- and \verb-Y-, elements 
(\verb-lelem-) with \verb-x-, and set lvalues (\verb-lset-) as \verb-Z-.

Note that the assignment and compound operators (\verb-= += |= &= &=~-)
are defined according to the pseudo-parallel semantics \cite{lois-sem}. That means, for each
valuation $v$ of the inner context of $Z$, the operation is performed on $Z_v$
in parallel for each valuation $w$ of the current context which extends $v$.
Also note that adding elements to a set is much more efficient than removing
(which basically loops over the set and keeps only the elements which are not
to be removed).

\vskip 1em

\begin{tabular}{ll}
\verb-X==Y-&set equality \\
\verb-X!=Y-&set inequality \\
\verb-X&Y-&set intersection \\
\verb-X&~Y-&set difference \\
\verb-X|Y-&set union \\
\verb-X*Y-&Cartesian product (\verb-elpair- used for the pairs)\\
\verb-cartesian({X,Y,Z,...})-&Cartesian product (\verb-eltuple- used for the tuples)\\
\verb-subseteq(X,Y)-&\verb-X- is a subset of \verb-Y- \\
\verb-memberof(X,x)-&\verb-x- is an element of \verb-X- \\
\verb-newSet()-&create an empty set\\
\verb-newSet(x)-&create the set $\{\verb-x-\}$\\
\verb-newSet(x,y)-&create the set $\{\verb-x-, \verb-y-\}$\\
\verb-newSet({x,y,...})-&create a set with the given elements\\
\verb-extract(X)-&extract the single element of a set\\
\verb-Z=X-&set \verb-Z- to \verb-X-\\
\verb-Z+=x-&add \verb-x- to the set \verb-Z-\\
\verb|Z-=x|&remove \verb-x- from the set \verb-Z- \\
\verb-Z|=X-&add all elements of \verb-X- to the set \verb-Z-\\
\verb-Z&=X-&remove all elements of \verb-Z- which are not elements of \verb-X-\\
\verb-Z&=~X-&remove all elements of \verb-Z- which are elements of \verb-X- \\
\end{tabular}

\subsection{The underlying structure}
To create an infinite set, construct an object of the class {\tt Domain};
this object's method {\tt getSet()} returns the underlying set, of type {\tt lset}.
It is possible to create multiple domains.

All domains are automatically equipped with equality ({\tt ==}, {\tt !=}) and 
a dense total order, accessed with the usual operators {\tt <}, {\tt >}, {\tt <=}, {\tt >=}.

LOIS supports domains which have more structure, for example, with two independent dense
total orders. This is explained in detail later (Section \ref{sec:relations}).

\subsection{Quantifier macros}
LOIS defines macros \verb-FORALL-, \verb-EXISTS-, \verb-FILTER-, \verb-MAP-, 
and \verb-FILTERMAP-, allowing
the programmer to construct formulas with quantifiers, and sets, intuitively. 
\verb-FILTER-$(x, A, \phi(x))$ corresponds to the mathematical set
$\{x | x\in A, \phi(x)\}$, \verb-MAP-$(x, A, v(x))$ corresponds to
$\{v(x) | x \in A\}$, and \verb-FILTERMAP-$(x, A, \phi(x), v(x))$ corresponds to
$\{v(x) | x\in A, \phi(x)\}$. All these macros are defined 
 using the for loop,
and are therefore redundant in terms of expressive power.

\subsection{Displaying the current context, and naming variables}
The variable \verb-currentcontext- of type \verb-contextptr- contains a pointer
to the current context on the stack, while \verb-emptycontext- is the pointer
to an empty stack. (See \cite{lois-sem} for the discussion of contexts and stacks.)
Use \verb-cout << c- to display the difference between the
\verb-currentcontext- and \verb-c-. Thus, the following will display $\inco{displaycontext}$:

\incc{displaycontext}

Note how we have named the variables occuring in terms \verb-a- and \verb-b- 
to $a$ and $b$ -- otherwise LOIS would not know how the programmer named them,
so it would generate its own names, which would be probably some random letters
instead of $a$ and $b$.

Other functions working with contexts include {\tt branchset(contextptr anccontext)}
(which returns the set of all pseudoparallel threads since {\tt anccontext}), and
{\tt getorbit} (see {\tt loisextra.h}), which returns the orbit of a given element
(see \cite{lois-sat}), treating the variables in the given ancestor context as fixed.

\subsection{Declaring atoms and axioms}

One can declare atoms and axioms, like in the following:
\incc{declareatom}

Create an object of type \verb-declareatom- to name one element of the domain.
Create an object of type \verb-axiom- to add an axiom, for example, that the
declared atoms are not equal (this is not given). This is implemented by
pushing the respective variables and constraints on the stack. A careful reader
will notice that the real effect is exactly the same as would be obtained by using
\verb-for(a1,A) for(a2,A) If(a1!=a2)-. However, when the programmer wants simply to
select some elements of $\bbA$, these constructs are more intuitive than loops and
conditionals.


% Thus, for example, the following outputs $\inco{quantifiers}$
% (which is not simplified in this case, because we asked to output the formula, not
% to evaluate it):
%
% \incc{quantifiers}
%
% And the following outputs $\inco{mapfilter}$:
%
% \incc{mapfilter}

\subsection{Choosing the solver}
The global variable {\tt solver} of type {\tt solveptr} describes the solver currently
used by LOIS. The following solvers, or solver combinations, are available:

\begin{itemize}
\item {\tt solverCrash()} which cannot solve anything (it just crashes).
\item {\tt solverBasic()} which can solve only the trivial cases.
\item {\tt solverExhaustive(int t, bool v)} calls the internal solver. The number 
{\tt t} corresponds to the number of tries (possible valuations) after which the
internal solver gives up, and {\tt v} is the verbosity.
\item {\tt solverSMT()}, {\tt solverCVC()} and {\tt solverSPASS()} call the given
external solver. They accept an optional {\tt std::string} argument, which is the path to the solver
executable.
\item {\tt solverIncremental(std::string)} uses Z3's incremental solving feature.
\item {\tt solverCompare(std::initializer\_list<solveptr> p)} compares the results of two or more solvers,
and checks for inconsistencies.
\item {\tt solverVerbose} and {\tt solverNamed(std::string n, solveptr s)} are wrappers 
around solvers which provide extra diagnostic information.
\item {\tt solverStack(solveptr s, solveptr fallback)} calls the solver {\tt s},
then calls {\tt fallback} if it failed. It can be used, for example, to solve simple cases with the
basic or exhaustive solver, then proceed to verbose exhaustive or external solver for
the harder cases.
\end{itemize}

Call {\tt useDefaultSolver(int i, int j)} to use the default solver (queries with
complexity below $i$ are solved in a non-verbose way, then they are solved in a verbose
way, and the internal solver gives up at $j$ tries; also $j$ is used as the limit for
the number of tries for the simplification algorithm).

\subsection{Other functions}
Some other functions include:
\begin{itemize}
\item {\tt lset optimize(const lset\& x)}, which optimizes the set by removing repetitions
(each element in the returned set will be in exactly one set-builder expression).
\item {\tt lset optimizeType(const lset\& x, const lset\& type)}, which is a more efficient version of
{\tt optimize} for the case when we know that {\tt x} is a subset of a (simple) set
{\tt type}.
\item {\tt getsingletonset(const lset\& X)}, which returns the set of singletons of elements of X,
represented as a single set-builder expression.

\end{itemize}

\section{Domains, symbols, and relations}\label{sec:relations}

It is possible to equip domains with extra structure, such as an order. LOIS includes
an internal solver for homogeneous with extension bounds \cite{lois-sat}; also,
several external solvers can be used for the structure $(\bbN, +)$ or $(\bbR, +, *)$.
Such an extra
structure does not have to be immediately declared when the domain is created; instead,
at any time the programmer can create an object from one of the subclasses of the class
{\tt Relation}, for example with the following declaration:

\begin{lstlisting}
RelOrder newOrder(" GT ", " LEQ ", " MAX ", " MIN ");
\end{lstlisting}

Terms representing elements of any domain can now be compared with respect to
{\tt newOrder} (a dense total order without endpoints) using the methods {\tt rbool less(const term\& a, const term\& b)} or
{\tt rbool leq(const term\& a, const term\& b)}. We can also find the maximum and minimum
of two elements, using the methods {\tt term max (const term\& a, const term\& b)}
and {\tt term min (const term\& a, const term\& b)}.

When the formulas are displayed on the screen, one of the symbols given in the
declaration of {\tt newOrder} is used. Symbols are given as objects of class
{\tt symbol}; a conversion from {\tt const char*} and {\tt std::string} to {\tt symbol}
is provided, but the class {\tt symbol} also allows to use symbols which are displayed
differently depending on the context. For example, the object {\tt sym} contains 
many symbols which are either used by LOIS itself or considered useful in applications,
and methods {\tt useUnicode()}, {\tt useLaTeX()},
and {\tt useASCII()}, which sets all the symbols to use the given format. The following
symbols are defined in the object {\tt sym}:

\begin{itemize}
\item Logical symbols:
{\tt exists}, {\tt forall}, {\tt \_and}, {\tt \_or}, {\tt eq}, {\tt neq}, {\tt in}

\item Basic set symbols:
{\tt emptyset}, {\tt ssunion} (an union of set-builder expessions), 
{\tt leftbrace}, {\tt rightbrace}, {\tt sfepipe} (a separator between the value
and the context in the set-builder expression), {\tt sfecomma} (a separator between
in the context of a set-builder expression), {\tt pseudo} (operator which extracts
a single element from a set, for the purpose of displaying {\tt lelem}s)

\item Relational and functional symbols:
{\tt leq}, {\tt geq}, {\tt greater}, {\tt less}, {\tt max}, {\tt min},
{\tt plus}, {\tt times}, {\tt minus}, {\tt divide}, {\tt edge}, {\tt noedge},
{\tt arrow}, {\tt noarrow}
\end{itemize}

LOIS declares the \emph{main order}, which is defined with the following line:

\begin{lstlisting}
  mainOrder = new RelOrder(sym.greater, sym.leq, sym.max, sym.min);
\end{lstlisting}

The C++ opperators {\tt <}, {\tt >}, {\tt <=} and {\tt >=}, when used on terms,
are defined as referring
to {\tt mainOrder}. When two orders (say, {\tt mainOrder} and {\tt newOrder}) are
defined and used in the same formula and over the same domain, they are 
considered to be unrelated -- intuitively, two random orders have been independently
chosen over our domain. In general, the same rule of independence is used when
defining multiple relations; in many cases, this yields oligomorphic homogeneous structures,
which still have a decidable first order theory \cite{lois-sat}. The algorithm used by
LOIS for deciding the satisfiability only takes into account relations which are actually
used in the given formula; so, the existence of {\tt mainOrder}, even if it is not used,
won't make the computations slower.

The following subclasses of {\tt Relation} are available:

\begin{itemize}
\item {\tt RelOrder}, explained above. 
The constructor of {\tt RelOrder} has
four symbol arguments: {\tt greater}, {\tt leq}, {\tt max}, and {\tt min}.
{\tt RelOrder} has four methods: {\tt less}, {\tt leq}, {\tt min}, and {\tt max}.

\item {\tt RelBinary}, which creates a random binary relation over the domain.
The constructor of {\tt RelOrder} has two symbol arguments {\tt inrel} (in the relation)
and {\tt notinrel} (not in the relation), as well as two parameter {\tt l} and {\tt s},
which define the properties of relation to be created. The parameter {\tt l} can take
the value of {\tt lmNoLoops} (irreflexive), {\tt lmAllLoops} (reflexive), or 
{\tt lmPossibleLoops} ($a$ can be in the relation with itself or not). The parameter
{\tt s} can take the value of {\tt smSymmetric}, {\tt smAsymmetric}, or {\tt smAntisymmetric}.
In all cases, with probability 1 we get the same graph
(up to isomorphism), which has a decidable first order theory due to being
homogeneous and oligomorphic (see \cite{lois-sat} for details); when choices
which leave more possibilities ({\tt lmPossibleLoops}, {\tt smAsymmetric})
are chosen, the internal solver has to consider all of them to verify the satisfiability
of formulas.
The relation is accessed with {\tt rbool operator () (const term\& a, const term\& b)}.

\item {\tt RelUnary}, which creates a random partition of the domain. The constructor is
{\tt RelUnary(symbol rel, int n)}, where every element of the domain is randomly
assigned to one of the {\tt n} parts. Operator {\tt rbool operator () (const term\& a, int v)}
is used to check whether {\tt a} is in the part number {\tt v} (0-based), and
for convenience, the method
{\tt rbool together(const term\& a, const term\& b)} checks whether {\tt a} and {\tt b}
are in the same parts.

\item {\tt RelTree}, which defines a homogeneous tree structure.
This is an infinite tree, where every two elements have the least common ancestor,
and it is dense without endpoints, that is, if $u$ is an ancestor of $v$, then
there is a $w$ such that $w$ is ancestor of $v$ and $u$ is ancestor of $w$, and
there is an ancestor of $u$ and a descendant of $w$; furthermore, there is infinite
branching, that is, if $v_1, \ldots, v_k$
are descendants of $u$, then there is $v$ such that the least common ancestor
of $v$ and $v_i$ for $i=1, \ldots, k$ is $u$. This structure showcases the fact
that, in a homogeneous structure with extension bound, the isomorphism type of
$\{a_1, \ldots, a_n\}$ might be defined not only by relations
on these elements -- we also need to check the relations of terms (in this case,
using lca).
% \cite{lois-sat}.
Methods {\tt anceq} (``ancestor or equal'') and {\tt lca} are used to
access the relations and functional symbols. The constructor has three symbol arguments
{\tt opanceq}, {\tt opnotanceq}, and {\tt oplca}, which correspond to the methods.

\item {\tt RelInt} and {\tt RelReal}, which define the set of integers and reals,
respectively. The constructor defines
the symbols related to the order, and also {\tt opplus}, {\tt optimes}, {\tt opminus},
and {\tt opdivide}. The term {\tt constant(Domain *d, int i)} is used for integer
constants, and there are also methods {\tt plus}, {\tt times}, {\tt minus}, and {\tt
divide} for the basic operations.
Note that these are not $\omega$-categorical,
and thus they are not compatible with other relations, and the internal solver does
not work with them -- an external solver is required \cite{lois-sat}.
Use the namespace {\tt orderedfield\_ops} from {\tt loisextra.h} to conveniently 
bind the C++ operators (\texttt{+ - * /}) to methods of the given {\tt RelInt} or
{\tt RelReal}.
\end{itemize}

Note that the domain for a given {\tt term} can be obtained with the method {\tt getDom()}.

\section{Piecewise v-elements} \label{app_piecewise}
The types \verb-lelem- and \verb-relem- %(see Appendix~\ref{app_pseudo})
 are useful for representing variables which behave in different ways under different valuations, e.g., $1|x=y;\ x|x\neq y.$

For a formal syntax of those types,  we  need the following notions.
A \emph{piecewise} v-element is an expression $e$ of the form
$e_1|C_1;e_2|C_2;\ldots; e_k|C_k,$
where $e_1,\ldots,e_k$ are v-elements, and $C_1,\ldots,C_k$ are constraints (i.e., first order formulas).
% Moreover, we assume that the constraints are mutually exclusive, i.e., there is no valuation satisfying two distinct constraints.
If $v$ is a valuation of the free variables of $e$,
then $e[v]$ is defined as $e_i[v]$, where $1\le i\le k$ is such that
$v$ satisfies $C_i$. If no such~$i$ exists, or it is not unique, then $e[v]$ is undefined.
% The set of free variables $V$ of $e$ is the union of the sets of free variables of $e_i$, for $i=1,\ldots,k$.
% Given a valuation $v$ of the free variables of $e$, define $e[v]$ to be equal to $e_i[v]$ if $\calA,v\models C_i$. In the case when there is no such $i$, $e[v]$ is undefined.
%OUT?

The types {\tt lelem} and {\tt relem} represent piecewise v-elements.
Similarly to {\tt lset}, {\tt lelem} is associated with an inner context.

\myparagraph{Assignments to {\tt lelem}}
In \cite{lois-sem}
%Section~\ref{sec_semantics} 
we only described how the assignments to variables of type {\tt lset} are carried out. A variable~{\tt x} of type {\tt lelem} designates a piecewise v-element.
 To see why using piecewise v-elements is necessary, consider the following example.

\incc{whypiecewise}
% Similarly to the previous example,  $Z$ should evaluate
%  $\set{x\st x\in \bbA,y\in \bbA, x=y}\cup \set{(x,y)\st x\in \bbA, y\in \bbA, x\neq y}$.
% In other words,
In order to guarantee the appropriate value of $\tt Z$ after executing this code, at the moment of the instruction \mbox{\tt Z+=u}, the variable~{\tt u}
needs to designate the piecewise v-element
$x|x=y; (x,y)|x\neq y.$

Internally, LOIS represents a variable {\tt x} of type {\tt lelem} designating the piecewise v-element $e_1|C_1;e_2|C_2;\ldots;e_k|C_k$
 by a variable \bra x of type {\tt lset}, which designates the v-set
 $\set{e_1|C_1}\cup\set{e_2|C_2}\ldots\cup\set{e_k|C_k}.$
 The v-set \bra x can be obtained in LOIS by the instruction {\tt newSet(x)}.
Operations on the type {\tt lelem} are carried out by lifting them to operations on the type {\tt lset}. For example,
the assignment {\tt x = y} to the variable {\tt x} of type {\tt lelem} is simulated by executing the assignment {\tt\bra x = newSet(y)}.

\myparagraph{New set}
If {\tt x} is of type {\tt elem}, then the instruction {\tt newSet(x)} is executed as expected-- it constructs a v-set of the form $\set{x}$. If {\tt x} is of type {\tt lelem}, this operation becomes slightly more involved, as described in the paragraph ``assignment'' above.


\myparagraph{Extract}
If {\tt X} is of type {\tt lset}, then the result of {\tt extract(X)} is of type {\tt relem} and  designates the piecewise v-element corresponding to the v-set $X$. On the level of implementation, the return value {\tt u} of this instruction is such that \bra u={\tt X}.

\begin{example}
To see the  usefulness of this operation, consider the following function which calculates the maximum of a set of terms,
such as $\{a_1, a_2\}$ or $\{x\st a_1 \leq x \leq a_2\}$. We assume here that  domain $\bbA$ is equipped with a linear order~$\ge$.

\incc{max}
First, we loop over all elements $x\in X$, and if $\forall y\in X. x \geq y$,
add $x$ to the set $answer$.
If the set $X$ indeed has a maximum, then $answer$
will be a singleton containing this maximum; otherwise; $answer$ will be empty.
If $X$ is equal to v-set $\{a_1, a_2\}$, then $answer$ will be calculated as $\{a_1| a_1\geq a_2; a_2|a_2\geq a_1\}$
and
% Now, we need to extract the only element of $answer$.
% Note that, for $M$,
% the only element is not a term. -- LOIS terms have a functional symbol for maximum,
% but LOIS is not intelligent enough to simplify it, and this approach would not
% work in general (for example, when the only element is a term in some pseudo-parallel
% branches, and in other ones it is a an integer constant or a set).
the
instruction {\tt extract(X)} will return the piecewise v-element
${a_1| a_1\geq a_2; a_2|a_2\geq a_1}$.
\end{example}

\subsection{Piecewise numbers} \label{app_piecewise_num}
The type \verb-lnum<T>- (where \verb-T- is usually \verb-int-, but
could be extended to other types) represents piecewise numbers. Operators are defined
for \verb-lnum<T>- according to the pseudo-parallel semantics;
in particular, \verb-x++- for each valuation $v$ of the internal context increments 
$x[v]$ by the cardinality of the set of possible valuations of the current context
which extend $v$. For example, consider the following program:

\incc{lnumtest}

If we know that $a \neq b$, this will output 2. If we don't know this, a representation
of the piecewise number will be generated, which currently is $$\inco{lnumtest}.$$ Note
the simplification algorithm failed to simplify the formula for the case when $a=b$.


% \section{Prover details}
%
% We have been trying to use external provers for these problems.
% The current prototype is able to export queries (empty theory only) to a
% smtlib-compliant SMT solver; it is not able to apply simplification tactics yet.
% We have tried using Yices \cite{yices}, CVC3 \cite{cvc3}, and Z3 \cite{z3},
% but our experiments so far did not yield satisfactory results:
% it seems that SMT solvers are not optimized for the kinds of formulas which are
% naturally generated
% by LOIS programs (first order formulas with many quantifiers), although this could
% be also caused by our lack of experience with SMT solvers.
% There are also first order provers based on the supposition calculus, such
% as SPASS \cite{spass}; however, SPASS did not yield satisfactory results either,
% and it does not seem to solve the problem of simplification.
% Another potential approach is to use QBF solvers -- but transforming a first order formula to QBF
% requires some work, and this does not solve the problem of simplification either.
%
% Instead, LOIS currently implements several simple techniques for simplification
% of formulas (eliminate quantifier when the value is known, eliminate repeating
% subformulas, etc.), and checks satisfiability by looping over all the possible
% assignments of variables, using the algorithm explained in Section
% \ref{sec_symbols}
% (with some optimizations --- that is, if the set of variables can
% be split into two subsets $V_1$ and $V_2$ such that we never compare variables
% in $V_1$ to variables in $V_2$, thet set of possible assignments is generated
% independently in $V_1$ and $V_2$ --- and relation is not generated for $V_i$
% if it is never used in this set). This is also occassionally done to simplify formulas (all the
% possible valuations are checked, and subformulas are replaced with {\tt true}
% or {\tt false} if a given subformula was always given the same result).
%

\section{Safety of programming}\label{subsecsafety}

It is possible to create a LOIS program which compiles and terminates without
a run-time exception, but nonetheless works incorrectly; for example, 
the following function will count the cardinality of the set incorrectly,
due to using the type \verb-int- instead of the piecewise integer \verb-lnum<int>-.

\incc{cardinalitybad}

Of course, this problem cannot be completely solved---it is possible to create an
incorrect C++ program even without LOIS.
However, the programmer should be mostly safe, if they observe the following
rules.

\begin{itemize}
\item The lvalue types ({\tt lbool}, {\tt lset}, {\tt lnum}, etc.) are used
for all the local (and global) variables, except the iterators for looping over
sets and 
\verb-setof<T>-'s, for which
\verb-elem- and \verb-T- (or \verb-auto-) are used respectively, and temporary values
(function
parameters, function return values), for which rvalue types may (and should) be used.
The difference between lvalue and rvalue types is technical---to handle
assignments and other changes correctly, lvalue types require extra
information (the inner context), while rvalue types do not.
We have decided to use two distinct types---the rationale is that 
a lvalue variable should correspond to a specific stack (context), and the stack
changes during a function call or return;
%(this rationale is not very
%strong, though, since only $V_S$ is important---not the whole stack---and
%$V_S$ does not change during function calls and returns)
also avoiding this
extra information could improve the efficiency.
Assigning a rvalue to a lvalue (or, in general, modifying a lvalue in any way
which takes a rvalue as a parameter) is legal only if the rvalue does not
depend on the sort variables which have been introduced since the
lvalue was created. Thus, the following implementation of
{\tt max} would throw an exception for some sets $X$:
\incc{maxbad}

\item Non-structural programming constructs, such as {\tt continue},
{\tt break}, {\tt return}, and {\tt goto} are considered harmful and
should be avoided. Suppose a function is called with the current context $C$,
then new constaints are pushed on the stack, changing the current context
to $C'$. Now, the {\tt return} statement is used. This will break the execution flow
not only for all the $C'$-valuations, but actually 
for all the $C$-valuations, irregardless of whether they are $C'$-valuations
or not (the necessary condition for executing 
the return statement is that $\val{C'}$ is non-empty).
Moreover, {\tt If} statements could execute both branches, and they are
internally implemented with a {\tt for} loop, which might make
the behavior of {\tt continue} and {\tt break} different than expected.
Thus, the only way of using the {\tt return} statement which is not considered
harmful is to  declare a local lvalue as the first statement (before any loops), 
modify it in the body of the function,
and then return its value as the last statement of the function.
Note that this approach is similar to the one used in original Pascal.
\end{itemize}




\section{Contents of the package}

The LOIS package included contains the following:

\begin{itemize}
\item source code (LOIS library itself, and the sample programs described above).
\item a Makefile which includes the {\tt all} target
which compiles the LOIS library and the
two programs, runs the two programs, and saves the output.
\item The result of
{\tt make all} (both binaries and their output), obtained on the first system.
\end{itemize}

%LOIS has been tested on machines with the following configurations:

%\begin{itemize}
%\item gcc version 4.4.3, architecture i686, Ubuntu Linux

%\item gcc version 4.6.3, architecture x86\_64, Ubuntu Linux
%\end{itemize}

%Note that, while all the core functionality of LOIS works, some simple useful operators
%and functions (in particular, those working on \verb|relem| as opposed to the simple
%\verb|elem|) might not be yet implemented, or might require some obvious type casting
%(a function variant for specific types might be missing, making C++ unable to guess
%what typecasts should be used).

\subsection{Sample programs}\label{app:samples}
The subdirectory \verb-tests- includes some sample programs. 

\subsubsection{Tutorial}

The program \verb-tests/tutorial.cpp- includes the tutorial given in 
Section \ref{sec_tut}, as well as some other code snippets quoted in this paper.
These snippets, as well as their outputs, are inserted
directly from \verb-tests/tutorial.cpp- or its output.
                                       
\subsubsection{Automatic tests}

The program \verb-tests/autotest.cpp- performs some automatic testing of LOIS.
This includes some interesting applications of LOIS, allowing
one to see that LOIS runs correctly, and how fast does it run.
The following tests are conducted:

\begin{itemize}
\item \verb-testRandomBipartite-

Let $R$ be a random symmetric
and anti-reflexive relation, and let $A_1$ and $A_2$ be a random 2-partition. Let
$S(x,y)$ iff $x$ and $y$ are in different parts of the partition.
We construct a new relation $E = R \cap S$. In graph theoretic terms,
$R$ is Rado's random graph, $S$ is a complete bipartite graph, and $E$ is a random
bipartite graph. We take one vertex $x \in \bbA$ and run BFS on the graph, and ask
about the number of iterations after which we have reached every vertex. The program
correctly answers that every vertex is reached after 3 iterations.

This test evaluates in roughly 3 milliseconds on the machine used for tests.

\item \verb-testTree-

A function is given elements
$x_1, \ldots, x_k$ of the homogeneous tree, and asks questions about relationships
between them. Once the answers uniquely determine the substructure generated by
$x_1, \ldots, x_k$, the substructure is presented in a readable form. For four elements
without any relations, 416 possible structures are generated (262 if we know that all
the four elements are not equal --- see sequences A005264 and A005172 in \cite{oeis}).

This test evaluates in roughly four seconds on our machine (for four elements).
This time is relatively long because of two reasons:

\begin{itemize}
\item 
The extension bound of the homogeneous
tree is relatively large ($e(n)=8n-4$, which gives the evaluation time of
roughly $8^k k!$ according to Proposition 1 in %\ref{prop:solver} in
\cite{lois-sat}). In fact 
Proposition 1 is not optimal, all trees are generated in time
roughly linear in the number of all trees, which is 416 for four elements. Still,
it grows quite fast.

\item The
program is very ineffective: currently, each question tries to generate all the possible
structures from the beginning, even if we know that some possible structures have been
already ruled out.
\end{itemize}

Therefore, the running time is actually at least quadratic in the
number of possible trees. This should be optimized in the future versions of LOIS.

\item \verb-testOrder-

This test the basic properties of the order relation, and evaluates very quickly.

\item \verb-testAssigment-

This checks whether an assignment exception is correctly thrown 
when we try to assign a value
(\verb-rbool-, in this case) which uses variables which are not in the internal
context of the variable we are assigning to.

\item \verb-testQueue-

This checks whether the \verb-setof-'s and the queue semantics of the \verb-for-
loop works correctly. Numbers from 0 to 10 are inserted to \verb-lsetof<int>-.

\item \verb-testRemoval-

This checks whether the \verb|-=| operator works in the natural, pseudo-parallel
way, as advertised in paper.
\end{itemize}

\subsubsection{Minimisation of an automaton}

The program \verb-tests/mintest.cpp- tries to perform the minimisation algorithm
on an orbit finite automaton. This automaton over the alphabet $\bbA$ (our infinite
domain) reads
three symbols, and accepts iff either two of them are equal (if there are less than
three or more than three symbols, the word is rejected). The minimisation algorithm
works in a way similar to the usual one for finite automata. The equivalence relation $\eta \subseteq Q \times Q$ is computed --
two states will be in $\eta$ if they can be merged into a single state. Initially,
$\eta$ is set to $F \times F \cup (Q-F) \times (Q-F)$, and then, in each iteration
states each $x,y \in Q$ are separated iff $\neg \eta(\delta(x,a), \delta(y,a))$ for some
symbol $a$ in the alphabet. For this particular automaton minimisation takes four
iterations. The algorithm is implemented using two representations ($\eta$ is
represented either as a relation or as the set of equivalence classes), and currently
takes 0.15 s in the relation representation, and 32 s in the equivalence class representation.

\subsubsection{Solver tests}\label{app:sol_tests}
The program \verb-tests/soltest.cpp- tests various solvers on several LOIS functions.
The table in \cite{lois-sat} is based on its results. The following tests
are included:

\begin{itemize}
\item \verb-testOrder- 
This test the basic properties of the order relation, and evaluates very quickly
with the internal solver, although external solvers have problems with it.

\item \verb-testReachable- 
Reachability from the article (Section \ref{sec_tut}).

\item \verb-testReal- 
This test the basic properties of the \verb-Real- sort (LRA logic).

\item \verb-testMinimize??-
These tests minimize automata. There are two automata: A (the same as in
\verb-tests/mintest.cpp-) and B (the automaton using the integers from the
introduction), and three different implementations of the minimisation algorithm
(two from \verb-tests/mintest.cpp-, and the one shown in the Introduction is
implementation number 3). Internal solver and SPASS work on the automaton A, but none of the solvers work on B.

\item \verb-testPacking-
What are the maximal sets of subsets of (0,5) such that no two points are in distance
less than 1? Z3 correctly calculates using the LRA logic that such maximal sets can
have from 3 to 5 elements.

\item \verb-testCirclePacking-
This tests the NRA logic by asking about packings of disks in a larger disk. 
None of the tested solvers can answer even the simplest questions here.
\end{itemize}

The external solvers where called using the following commands:


  \begin{description}\footnotesize
  \def\solver#1#2{\item [\bf #1:] {\tt #2}}
  \solver{Z3}{z3-*/bin/z3 -smt2 -in -t:500}
  \solver{CVC4}{cvc4 --lang smt --incremental --tlimit-per=500}
  \solver{CVC4*}{{cvc4 --lang smt --incremental --finite-model-find --tlimit-per=500}}
  \solver{SPASS}{SPASS -TimeLimit=1}
  \end{description}

\subsubsection{Learning automata}

The paper \cite{learningnominal} extends the famous Angluin algorithm of learning
of deterministic finite automata, as well as Bollig et al's version for non-deterministic
finite automata, to definable automata. The sample program {\tt tests/learning.cpp}
implements the algorithm given in \cite{learningnominal}, as well as includes tests
based on the paper. Multiple algorithms and tests can be activated by enabling and
disabling the conditional definitions at the head of the source code.

\subsubsection{Reducing definable CSPs to finite ones}

The paper \cite{infcsp} shows a reduction from definable CSPs over a finite template
to finite CSPs. This reduction is implemented by the program {\tt tests/csp.cpp}.


\def\whereislois{See \url{http://www.mimuw.edu.pl/~erykk/lois/}}
\bibliographystyle{alpha}
\bibliography{lois}{}


%\input{appendix}


\end{document}
